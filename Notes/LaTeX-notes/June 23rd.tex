\documentclass[10pt,oneside]{article}
%%%% Page Info + Commands %%%%%{

%packages
\usepackage{geometry}
\usepackage{latexsym}
\usepackage{amssymb}
\usepackage{amsfonts}
\usepackage{amstext}
\usepackage{amsmath}
\usepackage{amsthm}
\usepackage{multicol}
\usepackage{hyperref}
\usepackage{enumerate}
\usepackage{tikz}
\usepackage{enumitem}
\usepackage{xcolor}


\setlength{\footskip}{-5mm}

% a good babble textwidth is 5.75in
\newcommand{\babblewidth}{\setlength\textwidth{5.75in}}


% This will stretch out the page
\newcommand{\bigpage}{  \setlength \oddsidemargin{-.25in}
            \setlength \textwidth{6.75in}
            \setlength \topmargin{-1in}
            \setlength \textheight{9.75in}}


%This will shrink the page
\newcommand{\smallpage}{  \setlength \oddsidemargin{.5in}
            \setlength \textwidth{5in}
            \setlength \topmargin{0in}
            \setlength \textheight{9in}}

\newcommand{\separator}{\vglue .1in\hrule\vglue .1in}

\newcommand{\pause}{\vglue .1in\hrulefill {\tiny Pause here}\hrulefill \vglue .1in}

%%general stuff
\newcommand{\caret}{\textasciicircum}

%This will put a circle around something.
\newcommand*\circled[1]{\tikz[baseline=(char.base)]{
            \node[shape=circle,draw,inner sep=2pt] (char) {#1};}}


% Commands for abstract
\newcommand{\Z}{\mathbb{Z}}
\newcommand{\R}{\mathbb{R}}
\newcommand{\C}{\mathbb{C}}
\newcommand{\normal}{\triangleleft}
\newcommand{\Q}{\mathbb{Q}}
\newcommand{\F}{\mathbb{F}}
\newcommand{\N}{\mathbb{N}}
\newcommand{\K}{\mathbb{K}}
\newcommand{\aut}[1]{{\rm Aut}(#1)}
\newcommand{\Ker}{{\rm Ker}\,}
\newcommand{\im}{{\rm Im}\,}
\newcommand{\cyclic}[1]{\langle #1 \rangle}
\newcommand{\isom}{\cong}
\newcommand{\autc}[1]{{\rm Aut_c}(#1)}
\newcommand{\autsub}[2]{{\rm Aut}_{#1}(#2)}

\newcommand{\vp}{\vspace{0.15cm}\\}
\newcommand{\vpp}{\vspace{0.25cm}\\}
\newcommand{\vpn}{\vspace{0.05cm}\\}
\newcommand{\rmv}[1]{\,\backslash\{#1\}}
\newcommand{\rmvs}[1]{\,\backslash{#1}}
\newcommand{\md}[1]{\,\text{mod } #1}

%%%%%%%% command for graphics %%%%%%%%%%%%%
%}
\definecolor{darkgreen}{rgb}{0.0, 0.5, 0.0}
\definecolor{sasha}{rgb}{0.0, 0.5, 0.5}
\definecolor{marcus}{rgb}{0.7, 0.3, 0.3}
\definecolor{sam}{rgb}{0.2, 0.2, 0.8}


\begin{document}

$$
\begin{bmatrix}
    1 & 0 & 1\\
    0 & 1 & 1
\end{bmatrix}
$$

We can imagine this as the first element leaving x and z alone, and setting y to 0. Additionally, we can imagine the second one setting x to 0 and leaving y and z alone. We need to worry about getting enough elements from the kernel so we can generate the whole ring.\vpp
We have a set of powers that will be invariant no matter what:
\begin{itemize}
    \item $x^p$
    \item $y^p$
    \item $z^p$
\end{itemize}
The trick is to use this trick to lower exponents in the kernel. \\
We also know that in the kernel we can consider:
$$xyz^{p-1}$$
So we have that:
\begin{align*}
    g_0&: xz^{p-1}\\
    g_1&: xy^{p-1}
\end{align*}
If we look at our roots of unity, we see that the total powers of each group action will result in a total order of $p$, so our root of unity $\mu^p = 1$. \vpp
Because they are all multiples of this first invariant and we can generate the entire invariant ring. \vpp
CLAIM: If we look at the $x^2y^2z^{p-2}$, it will also be invariant because the total order will be $p$. 

\end{document}