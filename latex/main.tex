%********************************************%
%*       Generated from PreTeXt source      *%
%*       on 2025-05-17T01:47:14Z       *%
%*   A recent stable commit (2022-07-01):   *%
%* 6c761d3dba23af92cba35001c852aac04ae99a5f *%
%*                                          *%
%*         https://pretextbook.org          *%
%*                                          *%
%********************************************%
\documentclass[oneside,10pt,]{book}
%% Custom Preamble Entries, early (use latex.preamble.early)
%% Default LaTeX packages
%%   1.  always employed (or nearly so) for some purpose, or
%%   2.  a stylewriter may assume their presence
\usepackage{geometry}
%% Some aspects of the preamble are conditional,
%% the LaTeX engine is one such determinant
\usepackage{ifthen}
%% etoolbox has a variety of modern conveniences
\usepackage{etoolbox}
\usepackage{ifxetex,ifluatex}
%% Raster graphics inclusion
\usepackage{graphicx}
%% Color support, xcolor package
%% Always loaded, for: add/delete text, author tools
%% Here, since tcolorbox loads tikz, and tikz loads xcolor
\PassOptionsToPackage{dvipsnames,svgnames,table}{xcolor}
\usepackage{xcolor}
%% begin: defined colors, via xcolor package, for styling
%% end: defined colors, via xcolor package, for styling
%% Colored boxes, and much more, though mostly styling
%% skins library provides "enhanced" skin, employing tikzpicture
%% boxes may be configured as "breakable" or "unbreakable"
%% "raster" controls grids of boxes, aka side-by-side
\usepackage{tcolorbox}
\tcbuselibrary{skins}
\tcbuselibrary{breakable}
\tcbuselibrary{raster}
%% We load some "stock" tcolorbox styles that we use a lot
%% Placement here is provisional, there will be some color work also
%% First, black on white, no border, transparent, but no assumption about titles
\tcbset{ bwminimalstyle/.style={size=minimal, boxrule=-0.3pt, frame empty,
colback=white, colbacktitle=white, coltitle=black, opacityfill=0.0} }
%% Second, bold title, run-in to text/paragraph/heading
%% Space afterwards will be controlled by environment,
%% independent of constructions of the tcb title
%% Places \blocktitlefont onto many block titles
\tcbset{ runintitlestyle/.style={fonttitle=\blocktitlefont\upshape\bfseries, attach title to upper} }
%% Spacing prior to each exercise, anywhere
\tcbset{ exercisespacingstyle/.style={before skip={1.5ex plus 0.5ex}} }
%% Spacing prior to each block
\tcbset{ blockspacingstyle/.style={before skip={2.0ex plus 0.5ex}} }
%% xparse allows the construction of more robust commands,
%% this is a necessity for isolating styling and behavior
%% The tcolorbox library of the same name loads the base library
\tcbuselibrary{xparse}
%% The tcolorbox library loads TikZ, its calc package is generally useful,
%% and is necessary for some smaller documents that use partial tcolor boxes
%% See:  https://github.com/PreTeXtBook/pretext/issues/1624
\usetikzlibrary{calc}
%% We use some more exotic tcolorbox keys to restore indentation to parboxes
\tcbuselibrary{hooks}
%% Save default paragraph indentation and parskip for use later, when adjusting parboxes
\newlength{\normalparindent}
\newlength{\normalparskip}
\AtBeginDocument{\setlength{\normalparindent}{\parindent}}
\AtBeginDocument{\setlength{\normalparskip}{\parskip}}
\newcommand{\setparstyle}{\setlength{\parindent}{\normalparindent}\setlength{\parskip}{\normalparskip}}%% Hyperref should be here, but likes to be loaded late
%%
%% Inline math delimiters, \(, \), need to be robust
%% 2016-01-31:  latexrelease.sty  supersedes  fixltx2e.sty
%% If  latexrelease.sty  exists, bugfix is in kernel
%% If not, bugfix is in  fixltx2e.sty
%% See:  https://tug.org/TUGboat/tb36-3/tb114ltnews22.pdf
%% and read "Fewer fragile commands" in distribution's  latexchanges.pdf
\IfFileExists{latexrelease.sty}{}{\usepackage{fixltx2e}}
%% Footnote counters and part/chapter counters are manipulated
%% April 2018:  chngcntr  commands now integrated into the kernel,
%% but circa 2018/2019 the package would still try to redefine them,
%% so we need to do the work of loading conditionally for old kernels.
%% From version 1.1a,  chngcntr  should detect defintions made by LaTeX kernel.
\ifdefined\counterwithin
\else
    \usepackage{chngcntr}
\fi
%% Text height identically 9 inches, text width varies on point size
%% See Bringhurst 2.1.1 on measure for recommendations
%% 75 characters per line (count spaces, punctuation) is target
%% which is the upper limit of Bringhurst's recommendations
\geometry{letterpaper,total={340pt,9.0in}}
%% Custom Page Layout Adjustments (use publisher page-geometry entry)
%% This LaTeX file may be compiled with pdflatex, xelatex, or lualatex executables
%% LuaTeX is not explicitly supported, but we do accept additions from knowledgeable users
%% The conditional below provides  pdflatex  specific configuration last
%% begin: engine-specific capabilities
\ifthenelse{\boolean{xetex} \or \boolean{luatex}}{%
%% begin: xelatex and lualatex-specific default configuration
\ifxetex\usepackage{xltxtra}\fi
%% realscripts is the only part of xltxtra relevant to lualatex 
\ifluatex\usepackage{realscripts}\fi
%% end:   xelatex and lualatex-specific default configuration
}{
%% begin: pdflatex-specific default configuration
%% We assume a PreTeXt XML source file may have Unicode characters
%% and so we ask LaTeX to parse a UTF-8 encoded file
%% This may work well for accented characters in Western language,
%% but not with Greek, Asian languages, etc.
%% When this is not good enough, switch to the  xelatex  engine
%% where Unicode is better supported (encouraged, even)
\usepackage[utf8]{inputenc}
%% end: pdflatex-specific default configuration
}
%% end:   engine-specific capabilities
%%
%% Fonts.  Conditional on LaTex engine employed.
%% Default Text Font: The Latin Modern fonts are
%% "enhanced versions of the [original TeX] Computer Modern fonts."
%% We use them as the default text font for PreTeXt output.
%% Default Monospace font: Inconsolata (aka zi4)
%% Sponsored by TUG: http://levien.com/type/myfonts/inconsolata.html
%% Loaded for documents with intentional objects requiring monospace
%% See package documentation for excellent instructions
%% fontspec will work universally if we use filename to locate OTF files
%% Loads the "upquote" package as needed, so we don't have to
%% Upright quotes might come from the  textcomp  package, which we also use
%% We employ the shapely \ell to match Google Font version
%% pdflatex: "varl" package option produces shapely \ell
%% pdflatex: "var0" package option produces plain zero (not used)
%% pdflatex: "varqu" package option produces best upright quotes
%% xelatex,lualatex: add OTF StylisticSet 1 for shapely \ell
%% xelatex,lualatex: add OTF StylisticSet 2 for plain zero (not used)
%% xelatex,lualatex: add OTF StylisticSet 3 for upright quotes
%%
%% Automatic Font Control
%% Portions of a document, are, or may, be affected by defined commands
%% These are perhaps more flexible when using  xelatex  rather than  pdflatex
%% The following definitions are meant to be re-defined in a style, using \renewcommand
%% They are scoped when employed (in a TeX group), and so should not be defined with an argument
\newcommand{\divisionfont}{\relax}
\newcommand{\blocktitlefont}{\relax}
\newcommand{\contentsfont}{\relax}
\newcommand{\pagefont}{\relax}
\newcommand{\tabularfont}{\relax}
\newcommand{\xreffont}{\relax}
\newcommand{\titlepagefont}{\relax}
%%
\ifthenelse{\boolean{xetex} \or \boolean{luatex}}{%
%% begin: font setup and configuration for use with xelatex
%% Generally, xelatex is necessary for non-Western fonts
%% fontspec package provides extensive control of system fonts,
%% meaning *.otf (OpenType), and apparently *.ttf (TrueType)
%% that live *outside* your TeX/MF tree, and are controlled by your *system*
%% (it is possible that a TeX distribution will place fonts in a system location)
%%
%% The fontspec package is the best vehicle for using different fonts in  xelatex
%% So we load it always, no matter what a publisher or style might want
%%
\usepackage{fontspec}
%%
%% begin: xelatex main font ("font-xelatex-main" template)
%% Latin Modern Roman is the default font for xelatex and so is loaded with a TU encoding
%% *in the format* so we can't touch it, only perhaps adjust it later
%% in one of two ways (then known by NFSS names such as "lmr")
%% (1) via NFSS with font family names such as "lmr" and "lmss"
%% (2) via fontspec with commands like \setmainfont{Latin Modern Roman}
%% The latter requires the font to be known at the system-level by its font name,
%% but will give access to OTF font features through optional arguments
%% https://tex.stackexchange.com/questions/470008/
%% where-and-how-does-fontspec-sty-specify-the-default-font-latin-modern-roman
%% http://tex.stackexchange.com/questions/115321
%% /how-to-optimize-latin-modern-font-with-xelatex
%%
%% end:   xelatex main font ("font-xelatex-main" template)
%% begin: xelatex mono font ("font-xelatex-mono" template)
%% (conditional on non-trivial uses being present in source)
\IfFontExistsTF{Inconsolatazi4-Regular.otf}{}{\GenericError{}{The font "Inconsolatazi4-Regular.otf" requested by PreTeXt output processed by the  xelatex  executable is not available.  Either a file cannot be located in default locations via a filename, or a font is not known by its name as part of your system.}{Consult the PreTeXt Guide for help with LaTeX fonts, or perhaps try using  pdflatex  as a test.}{}}
\IfFontExistsTF{Inconsolatazi4-Bold.otf}{}{\GenericError{}{The font "Inconsolatazi4-Bold.otf" requested by PreTeXt output processed by the  xelatex  executable is not available.  Either a file cannot be located in default locations via a filename, or a font is not known by its name as part of your system.}{Consult the PreTeXt Guide for help with LaTeX fonts, or perhaps try using  pdflatex  as a test.}{}}
\usepackage{zi4}
\setmonofont[BoldFont=Inconsolatazi4-Bold.otf,StylisticSet={1,3}]{Inconsolatazi4-Regular.otf}
%% end:   xelatex mono font ("font-xelatex-mono" template)
%% begin: xelatex font adjustments ("font-xelatex-style" template)
%% end:   xelatex font adjustments ("font-xelatex-style" template)
%%
%% Extensive support for other languages
\usepackage{polyglossia}
%% Set main/default language based on pretext/@xml:lang value
%% document language code is "en-US", US English
%% usmax variant has extra hypenation
\setmainlanguage[variant=usmax]{english}
%% Enable secondary languages based on discovery of @xml:lang values
%% Enable fonts/scripts based on discovery of @xml:lang values
%% Western languages should be ably covered by Latin Modern Roman
%% end:   font setup and configuration for use with xelatex
}{%
%% begin: font setup and configuration for use with pdflatex
%% begin: pdflatex main font ("font-pdflatex-main" template)
\usepackage{lmodern}
\usepackage[T1]{fontenc}
%% end:   pdflatex main font ("font-pdflatex-main" template)
%% begin: pdflatex mono font ("font-pdflatex-mono" template)
%% (conditional on non-trivial uses being present in source)
\usepackage[varqu,varl]{inconsolata}
%% end:   pdflatex mono font ("font-pdflatex-mono" template)
%% begin: pdflatex font adjustments ("font-pdflatex-style" template)
%% end:   pdflatex font adjustments ("font-pdflatex-style" template)
%% end:   font setup and configuration for use with pdflatex
}
%% Micromanage spacing, etc.  The named "microtype-options"
%% template may be employed to fine-tune package behavior
\usepackage{microtype}
%% Symbols, align environment, commutative diagrams, bracket-matrix
\usepackage{amsmath}
\usepackage{amscd}
\usepackage{amssymb}
%% allow page breaks within display mathematics anywhere
%% level 4 is maximally permissive
%% this is exactly the opposite of AMSmath package philosophy
%% there are per-display, and per-equation options to control this
%% split, aligned, gathered, and alignedat are not affected
\allowdisplaybreaks[4]
%% allow more columns to a matrix
%% can make this even bigger by overriding with  latex.preamble.late  processing option
\setcounter{MaxMatrixCols}{30}
%%
%%
%% Division Titles, and Page Headers/Footers
%% titlesec package, loading "titleps" package cooperatively
%% See code comments about the necessity and purpose of "explicit" option.
%% The "newparttoc" option causes a consistent entry for parts in the ToC 
%% file, but it is only effective if there is a \titleformat for \part.
%% "pagestyles" loads the  titleps  package cooperatively.
\usepackage[explicit, newparttoc, pagestyles]{titlesec}
%% The companion titletoc package for the ToC.
\usepackage{titletoc}
%% Fixes a bug with transition from chapters to appendices in a "book"
%% See generating XSL code for more details about necessity
\newtitlemark{\chaptertitlename}
%% begin: customizations of page styles via the modal "titleps-style" template
%% Designed to use commands from the LaTeX "titleps" package
%% Plain pages should have the same font for page numbers
\renewpagestyle{plain}{%
\setfoot{}{\pagefont\thepage}{}%
}%
%% Single pages as in default LaTeX
\renewpagestyle{headings}{%
\sethead{\pagefont\slshape\MakeUppercase{\ifthechapter{\chaptertitlename\space\thechapter.\space}{}\chaptertitle}}{}{\pagefont\thepage}%
}%
\pagestyle{headings}
%% end: customizations of page styles via the modal "titleps-style" template
%%
%% Create globally-available macros to be provided for style writers
%% These are redefined for each occurence of each division
\newcommand{\divisionnameptx}{\relax}%
\newcommand{\titleptx}{\relax}%
\newcommand{\subtitleptx}{\relax}%
\newcommand{\shortitleptx}{\relax}%
\newcommand{\authorsptx}{\relax}%
\newcommand{\epigraphptx}{\relax}%
%% Create environments for possible occurences of each division
%% Environment for a PTX "chapter" at the level of a LaTeX "chapter"
\NewDocumentEnvironment{chapterptx}{mmmmmmm}
{%
\renewcommand{\divisionnameptx}{#1}%
\renewcommand{\titleptx}{#2}%
\renewcommand{\subtitleptx}{#3}%
\renewcommand{\shortitleptx}{#4}%
\renewcommand{\authorsptx}{#5}%
\renewcommand{\epigraphptx}{#6}%
\chapter[{#4}]{#2}%
\label{#7}%
}{}%
%% Environment for a PTX "section" at the level of a LaTeX "section"
\NewDocumentEnvironment{sectionptx}{mmmmmmm}
{%
\renewcommand{\divisionnameptx}{#1}%
\renewcommand{\titleptx}{#2}%
\renewcommand{\subtitleptx}{#3}%
\renewcommand{\shortitleptx}{#4}%
\renewcommand{\authorsptx}{#5}%
\renewcommand{\epigraphptx}{#6}%
\section[{#4}]{#2}%
\label{#7}%
}{}%
%% Environment for a PTX "subsection" at the level of a LaTeX "subsection"
\NewDocumentEnvironment{subsectionptx}{mmmmmmm}
{%
\renewcommand{\divisionnameptx}{#1}%
\renewcommand{\titleptx}{#2}%
\renewcommand{\subtitleptx}{#3}%
\renewcommand{\shortitleptx}{#4}%
\renewcommand{\authorsptx}{#5}%
\renewcommand{\epigraphptx}{#6}%
\subsection[{#4}]{#2}%
\label{#7}%
}{}%
%%
%% Styles for six traditional LaTeX divisions
\titleformat{\part}[display]
{\divisionfont\Huge\bfseries\centering}{\divisionnameptx\space\thepart}{30pt}{\Huge#1}
[{\Large\centering\authorsptx}]
\titleformat{\chapter}[display]
{\divisionfont\huge\bfseries}{\divisionnameptx\space\thechapter}{20pt}{\Huge#1}
[{\Large\authorsptx}]
\titleformat{name=\chapter,numberless}[display]
{\divisionfont\huge\bfseries}{}{0pt}{#1}
[{\Large\authorsptx}]
\titlespacing*{\chapter}{0pt}{50pt}{40pt}
\titleformat{\section}[hang]
{\divisionfont\Large\bfseries}{\thesection}{1ex}{#1}
[{\large\authorsptx}]
\titleformat{name=\section,numberless}[block]
{\divisionfont\Large\bfseries}{}{0pt}{#1}
[{\large\authorsptx}]
\titlespacing*{\section}{0pt}{3.5ex plus 1ex minus .2ex}{2.3ex plus .2ex}
\titleformat{\subsection}[hang]
{\divisionfont\large\bfseries}{\thesubsection}{1ex}{#1}
[{\normalsize\authorsptx}]
\titleformat{name=\subsection,numberless}[block]
{\divisionfont\large\bfseries}{}{0pt}{#1}
[{\normalsize\authorsptx}]
\titlespacing*{\subsection}{0pt}{3.25ex plus 1ex minus .2ex}{1.5ex plus .2ex}
\titleformat{\subsubsection}[hang]
{\divisionfont\normalsize\bfseries}{\thesubsubsection}{1em}{#1}
[{\small\authorsptx}]
\titleformat{name=\subsubsection,numberless}[block]
{\divisionfont\normalsize\bfseries}{}{0pt}{#1}
[{\normalsize\authorsptx}]
\titlespacing*{\subsubsection}{0pt}{3.25ex plus 1ex minus .2ex}{1.5ex plus .2ex}
\titleformat{\paragraph}[hang]
{\divisionfont\normalsize\bfseries}{\theparagraph}{1em}{#1}
[{\small\authorsptx}]
\titleformat{name=\paragraph,numberless}[block]
{\divisionfont\normalsize\bfseries}{}{0pt}{#1}
[{\normalsize\authorsptx}]
\titlespacing*{\paragraph}{0pt}{3.25ex plus 1ex minus .2ex}{1.5em}
%%
%% Styles for five traditional LaTeX divisions
\titlecontents{part}%
[0pt]{\contentsmargin{0em}\addvspace{1pc}\contentsfont\bfseries}%
{\Large\thecontentslabel\enspace}{\Large}%
{}%
[\addvspace{.5pc}]%
\titlecontents{chapter}%
[0pt]{\contentsmargin{0em}\addvspace{1pc}\contentsfont\bfseries}%
{\large\thecontentslabel\enspace}{\large}%
{\hfill\bfseries\thecontentspage}%
[\addvspace{.5pc}]%
\dottedcontents{section}[3.8em]{\contentsfont}{2.3em}{1pc}%
\dottedcontents{subsection}[6.1em]{\contentsfont}{3.2em}{1pc}%
\dottedcontents{subsubsection}[9.3em]{\contentsfont}{4.3em}{1pc}%
%%
%% Begin: Semantic Macros
%% To preserve meaning in a LaTeX file
%%
%% \mono macro for content of "c", "cd", "tag", etc elements
%% Also used automatically in other constructions
%% Simply an alias for \texttt
%% Always defined, even if there is no need, or if a specific tt font is not loaded
\newcommand{\mono}[1]{\texttt{#1}}
%%
%% Following semantic macros are only defined here if their
%% use is required only in this specific document
%%
%% Used for inline definitions of terms
\newcommand{\terminology}[1]{\textbf{#1}}
%% End: Semantic Macros
%% Footnote Numbering
%% Specified by numbering.footnotes.level
%% Undo counter reset by chapter for a book
\counterwithout{footnote}{chapter}
\counterwithin*{footnote}{section}
%% Program listing support: for listings, programs, consoles, and Sage code
\ifthenelse{\boolean{xetex} \or \boolean{luatex}}%
  {\tcbuselibrary{listings}}%
  {\tcbuselibrary{listingsutf8}}%
%% We define the listings font style to be the default "ttfamily"
%% To fix hyphens/dashes rendered in PDF as fancy minus signs by listing
%% http://tex.stackexchange.com/questions/33185/listings-package-changes-hyphens-to-minus-signs
\makeatletter
\lst@CCPutMacro\lst@ProcessOther {"2D}{\lst@ttfamily{-{}}{-{}}}
\@empty\z@\@empty
\makeatother
%% We define a null language, free of any formatting or style
%% for use when a language is not supported, or pseudo-code, or consoles
%% Not necessary for Sage code, so in limited cases included unnecessarily
\lstdefinelanguage{none}{identifierstyle=,commentstyle=,stringstyle=,keywordstyle=}
\ifthenelse{\boolean{xetex}}{}{%
%% begin: pdflatex-specific listings configuration
%% translate U+0080 - U+00F0 to their textmode LaTeX equivalents
%% Data originally from https://www.w3.org/Math/characters/unicode.xml, 2016-07-23
%% Lines marked in XSL with "$" were converted from mathmode to textmode
\lstset{extendedchars=true}
\lstset{literate={ }{{~}}{1}{¡}{{\textexclamdown }}{1}{¢}{{\textcent }}{1}{£}{{\textsterling }}{1}{¤}{{\textcurrency }}{1}{¥}{{\textyen }}{1}{¦}{{\textbrokenbar }}{1}{§}{{\textsection }}{1}{¨}{{\textasciidieresis }}{1}{©}{{\textcopyright }}{1}{ª}{{\textordfeminine }}{1}{«}{{\guillemotleft }}{1}{¬}{{\textlnot }}{1}{­}{{\-}}{1}{®}{{\textregistered }}{1}{¯}{{\textasciimacron }}{1}{°}{{\textdegree }}{1}{±}{{\textpm }}{1}{²}{{\texttwosuperior }}{1}{³}{{\textthreesuperior }}{1}{´}{{\textasciiacute }}{1}{µ}{{\textmu }}{1}{¶}{{\textparagraph }}{1}{·}{{\textperiodcentered }}{1}{¸}{{\c{}}}{1}{¹}{{\textonesuperior }}{1}{º}{{\textordmasculine }}{1}{»}{{\guillemotright }}{1}{¼}{{\textonequarter }}{1}{½}{{\textonehalf }}{1}{¾}{{\textthreequarters }}{1}{¿}{{\textquestiondown }}{1}{À}{{\`{A}}}{1}{Á}{{\'{A}}}{1}{Â}{{\^{A}}}{1}{Ã}{{\~{A}}}{1}{Ä}{{\"{A}}}{1}{Å}{{\AA }}{1}{Æ}{{\AE }}{1}{Ç}{{\c{C}}}{1}{È}{{\`{E}}}{1}{É}{{\'{E}}}{1}{Ê}{{\^{E}}}{1}{Ë}{{\"{E}}}{1}{Ì}{{\`{I}}}{1}{Í}{{\'{I}}}{1}{Î}{{\^{I}}}{1}{Ï}{{\"{I}}}{1}{Ð}{{\DH }}{1}{Ñ}{{\~{N}}}{1}{Ò}{{\`{O}}}{1}{Ó}{{\'{O}}}{1}{Ô}{{\^{O}}}{1}{Õ}{{\~{O}}}{1}{Ö}{{\"{O}}}{1}{×}{{\texttimes }}{1}{Ø}{{\O }}{1}{Ù}{{\`{U}}}{1}{Ú}{{\'{U}}}{1}{Û}{{\^{U}}}{1}{Ü}{{\"{U}}}{1}{Ý}{{\'{Y}}}{1}{Þ}{{\TH }}{1}{ß}{{\ss }}{1}{à}{{\`{a}}}{1}{á}{{\'{a}}}{1}{â}{{\^{a}}}{1}{ã}{{\~{a}}}{1}{ä}{{\"{a}}}{1}{å}{{\aa }}{1}{æ}{{\ae }}{1}{ç}{{\c{c}}}{1}{è}{{\`{e}}}{1}{é}{{\'{e}}}{1}{ê}{{\^{e}}}{1}{ë}{{\"{e}}}{1}{ì}{{\`{\i}}}{1}{í}{{\'{\i}}}{1}{î}{{\^{\i}}}{1}{ï}{{\"{\i}}}{1}{ð}{{\dh }}{1}{ñ}{{\~{n}}}{1}{ò}{{\`{o}}}{1}{ó}{{\'{o}}}{1}{ô}{{\^{o}}}{1}{õ}{{\~{o}}}{1}{ö}{{\"{o}}}{1}{÷}{{\textdiv }}{1}{ø}{{\o }}{1}{ù}{{\`{u}}}{1}{ú}{{\'{u}}}{1}{û}{{\^{u}}}{1}{ü}{{\"{u}}}{1}{ý}{{\'{y}}}{1}{þ}{{\th }}{1}{ÿ}{{\"{y}}}{1}}
%% end: pdflatex-specific listings configuration
}
%% End of generic listing adjustments
%% The listings package as tcolorbox for Sage code
%% We do as much styling as possible with tcolorbox, not listings
%% Sage's blue is 50%, we go way lighter (blue!05 would also work)
%% Note that we defuse listings' default "aboveskip" and "belowskip"
\definecolor{sageblue}{rgb}{0.95,0.95,1}
\tcbset{ sagestyle/.style={left=0pt, right=0pt, top=0ex, bottom=0ex, middle=0pt, toptitle=0pt, bottomtitle=0pt,
boxsep=4pt, listing only, fontupper=\small\ttfamily,
breakable, 
listing options={language=Python,breaklines=true,breakatwhitespace=true, extendedchars=true, aboveskip=0pt, belowskip=0pt}} }
\newtcblisting{sageinput}{sagestyle, colback=sageblue, sharp corners, boxrule=0.5pt, toprule at break=-0.3pt, bottomrule at break=-0.3pt, }
\newtcblisting{sageoutput}{sagestyle, colback=white, colframe=white, frame empty, before skip=0pt, after skip=0pt, }
%% More flexible list management, esp. for references
%% But also for specifying labels (i.e. custom order) on nested lists
\usepackage{enumitem}
%% hyperref driver does not need to be specified, it will be detected
%% Footnote marks in tcolorbox have broken linking under
%% hyperref, so it is necessary to turn off all linking
%% It *must* be given as a package option, not with \hypersetup
\usepackage[hyperfootnotes=false]{hyperref}
%% configure hyperref's  \href{}{}  and  \nolinkurl  to match listings' inline verbatim
\renewcommand\UrlFont{\small\ttfamily}
%% Hyperlinking active in electronic PDFs, all links without surrounding boxes and blue
\hypersetup{colorlinks=true,linkcolor=blue,citecolor=blue,filecolor=blue,urlcolor=blue}
%% Less-clever names for hyperlinks are more reliable, *especially* for structural parts
%% See comments in the code to learn more about the importance of this setting
\hypersetup{hypertexnames=false}
%%The  hypertexnames  setting then confuses the hyperlinking from the index
%%This patch resolves the incorrect links, see code for StackExchange post.
\makeatletter
\patchcmd\Hy@EveryPageBoxHook{\Hy@EveryPageAnchor}{\Hy@hypertexnamestrue\Hy@EveryPageAnchor}{}{\fail}
\makeatother
\hypersetup{pdftitle={Invariant Rings in Macaulay2}}
%% If you manually remove hyperref, leave in this next command
%% This will allow LaTeX compilation, employing this no-op command
\providecommand\phantomsection{}
%% Division Numbering: Chapters, Sections, Subsections, etc
%% Division numbers may be turned off at some level ("depth")
%% A section *always* has depth 1, contrary to us counting from the document root
%% The latex default is 3.  If a larger number is present here, then
%% removing this command may make some cross-references ambiguous
%% The precursor variable $numbering-maxlevel is checked for consistency in the common XSL file
\setcounter{secnumdepth}{3}
%%
%%
%% A faux tcolorbox whose only purpose is to provide common numbering
%% facilities for most blocks (possibly not projects, 2D displays)
%% Controlled by  numbering.theorems.level  processing parameter
\newtcolorbox[auto counter, number within=section]{block}{}
%%
%% This document is set to number PROJECT-LIKE on a separate numbering scheme
%% So, a faux tcolorbox whose only purpose is to provide this numbering
%% Controlled by  numbering.projects.level  processing parameter
\newtcolorbox[auto counter, number within=section]{project-distinct}{}
%% A faux tcolorbox whose only purpose is to provide common numbering
%% facilities for 2D displays which are subnumbered as part of a "sidebyside"
\makeatletter
\newtcolorbox[auto counter, number within=tcb@cnt@block, number freestyle={\noexpand\thetcb@cnt@block(\noexpand\alph{\tcbcounter})}]{subdisplay}{}
\makeatother
%%
%% tcolorbox, with styles, for THEOREM-LIKE
%%
%% theorem: fairly simple numbered block/structure
\tcbset{ theoremstyle/.style={bwminimalstyle, runintitlestyle, blockspacingstyle, after title={\space}, before upper app={\setparstyle}, } }
\newtcolorbox[use counter from=block]{theorem}[4]{title={{#1~\thetcbcounter\notblank{#2#3}{\space}{}\notblank{#2}{\space#2}{}\notblank{#3}{\space(#3)}{}}}, phantomlabel={#4}, breakable, after={\par}, fontupper=\itshape, theoremstyle, }
%% corollary: fairly simple numbered block/structure
\tcbset{ corollarystyle/.style={bwminimalstyle, runintitlestyle, blockspacingstyle, after title={\space}, before upper app={\setparstyle}, } }
\newtcolorbox[use counter from=block]{corollary}[4]{title={{#1~\thetcbcounter\notblank{#2#3}{\space}{}\notblank{#2}{\space#2}{}\notblank{#3}{\space(#3)}{}}}, phantomlabel={#4}, breakable, after={\par}, fontupper=\itshape, corollarystyle, }
%% proposition: fairly simple numbered block/structure
\tcbset{ propositionstyle/.style={bwminimalstyle, runintitlestyle, blockspacingstyle, after title={\space}, before upper app={\setparstyle}, } }
\newtcolorbox[use counter from=block]{proposition}[4]{title={{#1~\thetcbcounter\notblank{#2#3}{\space}{}\notblank{#2}{\space#2}{}\notblank{#3}{\space(#3)}{}}}, phantomlabel={#4}, breakable, after={\par}, fontupper=\itshape, propositionstyle, }
%% algorithm: fairly simple numbered block/structure
\tcbset{ algorithmstyle/.style={bwminimalstyle, runintitlestyle, blockspacingstyle, after title={\space}, before upper app={\setparstyle}, } }
\newtcolorbox[use counter from=block]{algorithm}[4]{title={{#1~\thetcbcounter\notblank{#2#3}{\space}{}\notblank{#2}{\space#2}{}\notblank{#3}{\space(#3)}{}}}, phantomlabel={#4}, breakable, after={\par}, fontupper=\itshape, algorithmstyle, }
%%
%% tcolorbox, with styles, for DEFINITION-LIKE
%%
%% definition: fairly simple numbered block/structure
\tcbset{ definitionstyle/.style={bwminimalstyle, runintitlestyle, blockspacingstyle, after title={\space}, after upper={\space\space\hspace*{\stretch{1}}\(\lozenge\)}, before upper app={\setparstyle}, } }
\newtcolorbox[use counter from=block]{definition}[3]{title={{#1~\thetcbcounter\notblank{#2}{\space\space#2}{}}}, phantomlabel={#3}, breakable, after={\par}, definitionstyle, }
%%
%% tcolorbox, with styles, for EXAMPLE-LIKE
%%
%% example: fairly simple numbered block/structure
\tcbset{ examplestyle/.style={bwminimalstyle, runintitlestyle, blockspacingstyle, after title={\space}, after upper={\space\space\hspace*{\stretch{1}}\(\square\)}, before upper app={\setparstyle}, } }
\newtcolorbox[use counter from=block]{example}[3]{title={{#1~\thetcbcounter\notblank{#2}{\space\space#2}{}}}, phantomlabel={#3}, breakable, after={\par}, examplestyle, }
%%
%% xparse environments for introductions and conclusions of divisions
%%
%% introduction: in a structured division
\NewDocumentEnvironment{introduction}{m}
{\notblank{#1}{\noindent\textbf{#1}\space}{}}{\par\medskip}
%% Graphics Preamble Entries
\usepackage{tikz}
%% If tikz has been loaded, replace ampersand with \amp macro
%% Custom Preamble Entries, late (use latex.preamble.late)
%% extpfeil package for certain extensible arrows,
%% as also provided by MathJax extension of the same name
%% NB: this package loads mtools, which loads calc, which redefines
%%     \setlength, so it can be removed if it seems to be in the 
%%     way and your math does not use:
%%     
%%     \xtwoheadrightarrow, \xtwoheadleftarrow, \xmapsto, \xlongequal, \xtofrom
%%     
%%     we have had to be extra careful with variable thickness
%%     lines in tables, and so also load this package late
\usepackage{extpfeil}
%% Begin: Author-provided macros
%% (From  docinfo/macros  element)
%% Plus three from PTX for XML characters
\newcommand{\R}{\mathbb R}
\newcommand{\lt}{<}
\newcommand{\gt}{>}
\newcommand{\amp}{&}
%% End: Author-provided macros
\begin{document}
%% bottom alignment is explicit, since it normally depends on oneside, twoside
\raggedbottom
%
%
\typeout{************************************************}
\typeout{Chapter 1 Invariant Theory}
\typeout{************************************************}
%
\begin{chapterptx}{Chapter}{Invariant Theory}{}{Invariant Theory}{}{}{ch-invarianttheory}
\renewcommand*{\chaptername}{Chapter}
\begin{introduction}{}%
This chapter is co-authored by Francesca Gandini, Al Ashir Intisar, and Sumner Strom. In this chapter we will present an overview of the theory behind the algorithms implemented in the \href{https://www.macaulay2.com/doc/Macaulay2/share/doc/Macaulay2/InvariantRing/html/index.html}{InvariantRing}\footnote{\nolinkurl{www.macaulay2.com/doc/Macaulay2/share/doc/Macaulay2/InvariantRing/html/index.html}\label{ch-invarianttheory-2-1-2}} software package in the open-source Computer Algebra System \href{http://www2.macaulay2.com}{Macaulay2 (M2)}\footnote{\nolinkurl{www2.macaulay2.com}\label{ch-invarianttheory-2-1-4}}.%
\par
You can access an online version of this chapter with live code cell at \href{https://fragandi.github.io/CURITutorialDevelopment2025/}{\nolinkurl{https://fragandi.github.io/CURITutorialDevelopment2025/}}. There you can also learn how to set up a virtual machine on Github with Codespaces so that you write and run M2 code from anywhere. A turn-key repository for creating a Codespace  for Macaulay2 is available at \href{https://github.com/fragandi/M2-codespace}{fragandi\slash{}M2-codespace}\footnote{\nolinkurl{github.com/fragandi/M2-codespace}\label{ch-invarianttheory-2-2-3}}.%
\par
We also include some background on orbit sums necessary to implement an algorithm to compute invariants for permutations actions. We have worked with a group of collaborators on the first version of the code for this algorithm at the Macaulay2 Workshop at Tulane University in April 2025 and plan to further test it and release it with Macaulay2 in Fall 2025.%
\par
We finish the chapter with a selection of examples that illustrate the current capabilities of the InvariantRing package. You can run the provided code in your local installation of M2 or go to the online version and execute the code cells on your browser. This works  well even on mobile devices!%
\end{introduction}%
%
%
\typeout{************************************************}
\typeout{Section 1.1 A concrete introduction to invariant rings}
\typeout{************************************************}
%
\begin{sectionptx}{Section}{A concrete introduction to invariant rings}{}{A concrete introduction to invariant rings}{}{}{sec-invariantrings}
%
%
\typeout{************************************************}
\typeout{Subsection 1.1.1 Finite Matrix Groups}
\typeout{************************************************}
%
\begin{subsectionptx}{Subsection}{Finite Matrix Groups}{}{Finite Matrix Groups}{}{}{subsec-finite-matrix-groups}
We can think of a (linear) action of a group on a vector space concretely by interpreting each group element as a matrix and describing the action as matrix multiplication on vectors. We can then evaluate any polynomial on a vector and its image after the action.%
\begin{example}{Example}{}{subsec-finite-matrix-groups-3}%
Consider%
\begin{equation*}
M =  \begin{pmatrix}
1 \amp 0 \\
0 \amp -1 \\
\end{pmatrix} 
\end{equation*}
and the vector \(\bar x = \begin{pmatrix} x\\ y\\ \end{pmatrix}\) This gives \(M \bar x = \begin{pmatrix}
x \\
-y  \\
\end{pmatrix}\). Thus for the polynomial%
\begin{equation*}
f(\bar x) = f(\begin{pmatrix}
x \\
y  \\
\end{pmatrix}) = x+y
\end{equation*}
and we have,%
\begin{equation*}
f(M\bar x) = f(\begin{pmatrix}
x \\
-y  \\
\end{pmatrix})= x-y\text{.}
\end{equation*}
%
\end{example}
More formally, for \(G \) a finite group we will consider a linear representation of \(G \) via its action on a finite dimensional vector space \(V \) over a field \(K \) of characteristics zero. In general, most of the results in this chapter hold in the non-modular case i.e., when the characteristics of the field does not divide the order of the group. As of now finite fields are not fully supported by the current version of the InvariantRing package and such functionalities is an active area of development.%
\par
If \(V \) is faithful representation of \(G \) of dimension \(m\), the image of the representation is isomorphic to \(G \) and so we consider \(G \) as a finite \terminology{matrix group}.%
\begin{definition}{Definition}{}{subsec-finite-matrix-groups-6}%
Suppose \(|G| < \infty\) and \(G \leq GL_m(\mathbb{K})\),  then \(G\) is a finite matrix group.%
\end{definition}
\begin{example}{Example}{}{subsec-finite-matrix-groups-7}%
Let us consider a two-dimesional representation of \(C_2\), the cyclic group of order 2.%
\begin{equation*}
\left\langle \begin{pmatrix}
1 \amp 0 \\
0 \amp -1 \\
\end{pmatrix} \right\rangle = \left\{ \begin{pmatrix}
1 \amp 0 \\
0 \amp -1 \\
\end{pmatrix},\begin{pmatrix}
1 \amp 0 \\
0 \amp 1 \\
\end{pmatrix} \right \} \cong C_2
\end{equation*}
%
\end{example}
\end{subsectionptx}
%
%
\typeout{************************************************}
\typeout{Subsection 1.1.2 Invariant Rings}
\typeout{************************************************}
%
\begin{subsectionptx}{Subsection}{Invariant Rings}{}{Invariant Rings}{}{}{subsec-invariant-rings}
We will work with a polynomial ring in \(n\) variables over the field \(\mathbb{K}\). We use the notation \(\bar x = (x_1, x_2,..., x_n)\) and abuse it by saying \(\mathbb{K}[x_1,x_2,...,x_n]=\mathbb{K}[\bar x]\) and \(f(x_1,x_2,...,x_n)=f(\bar x)\) for \(f \in \mathbb{K}[\bar x]\).%
\begin{definition}{Definition}{}{subsec-invariant-rings-3}%
Let \(G\) be a finite matrix group within \(GL_n(\mathbb{K})\). We say that \(f\in \mathbb{K}[\bar x]\) is invariant under the action of \(G\) if and only if%
\begin{equation*}
f(A\bar x) = f(\bar x),
\end{equation*}
for all \(A \in G\).%
\end{definition}
\begin{example}{Example}{}{subsec-invariant-rings-4}%
The polynomials \(f(\bar x)=x\) and \(f(\bar x) = x +y^2\) in \(\mathbb{K}[x,y]\) are invariant under the action of%
\begin{equation*}
C_2 = \left\langle\begin{pmatrix}
1 \amp 0 \\
0 \amp -1 \\
\end{pmatrix} \right\rangle
\end{equation*}
However the polynoial \(f(\bar x)=x+y\) is not.%
\end{example}
We can consider the set of all invariant polynomials under some action of a group \(G \). A good exercise is to prove that this set has the structure of a ring.%
\begin{definition}{Definition}{}{subsec-invariant-rings-6}%
Let \(R= \mathbb{K}[\bar x]\). We define%
\begin{equation*}
R^G : =  \{f \in R \, | f(A\bar x) = f(\bar x), \, \forall A \in G\} \subseteq R
\end{equation*}
to be the invariant ring for the action of \(G\) on \(R\).%
\end{definition}
\end{subsectionptx}
%
%
\typeout{************************************************}
\typeout{Subsection 1.1.3 Reynolds Operator}
\typeout{************************************************}
%
\begin{subsectionptx}{Subsection}{Reynolds Operator}{}{Reynolds Operator}{}{}{subsec-reynolds-operator}
We have that the invariant ring \(R^G\) is a subring of the ring \(R= \mathbb{K}[\bar x]\). However, it is not an ordinary subring. In characteristic zero, we have a \terminology{projection} from \(R\) to \(R^G\) that respects the action of \(G\). The idea: "averaging" over the action of \(G\) we get an invariant polynomial.%
\begin{definition}{Definition}{}{subsec-reynolds-operator-3}%
The averaging (or Reynolds) map \(R_G: R \xrightarrow{} R^G\) is given by%
\begin{equation*}
R_G(f) = \frac{1}{|G|} \sum_{A\in G} f(A \bar x) 
\end{equation*}
%
\end{definition}
\begin{example}{Example}{}{subsec-reynolds-operator-4}%
Example for the Group action \(C_2 = \left\langle\begin{pmatrix}
1 \amp 0 \\
0 \amp -1 \\
\end{pmatrix}\right\rangle\). Consider the polynomial \(x+y\), which is not invariant under the action of \(C_2\). We have that:%
\begin{equation*}
R_G(x+y) = \frac{1}{2} ((x+y) + (x-y)) = x\in R^G
\end{equation*}
and we can check that \(R_G(x+y)=x\) is indeed invariant.%
\end{example}
\end{subsectionptx}
\end{sectionptx}
%
%
\typeout{************************************************}
\typeout{Section 1.2 Degree bounds and algorithms}
\typeout{************************************************}
%
\begin{sectionptx}{Section}{Degree bounds and algorithms}{}{Degree bounds and algorithms}{}{}{sec-degree-bounds-algorithms}
\begin{introduction}{}%
Our goal is to find algorithms that provide us with a description of all possible invariants in an efficient way. Formally, we look for \terminology{minimal generators} for the ring of invariants \(R^G\) and more precisely for minimal algebra generators for \(R^G\) as an algebra over the coefficient field \(\mathbb{K}\).%
\par
For our search to be successful, we need to hope that there are finitely many generators. In our setup (finite groups and characteristic zero) a consequence of Hilbert's Basis Theorem is that our invariant rings are finitely generated. However, we will run in computational troubles if we do not have a stopping point for our search. The most effective way is to provide a bound the degrees of these generators.%
\end{introduction}%
%
%
\typeout{************************************************}
\typeout{Subsection 1.2.1 Noether Degree Bound}
\typeout{************************************************}
%
\begin{subsectionptx}{Subsection}{Noether Degree Bound}{}{Noether Degree Bound}{}{}{subsec-noether-degree-bound}
A beautiful theorem of Noether establishes that we have a bound on the degree of a minimal generator independent from the action itself, but just in terms of the order of the group. Moreover, we only need to look at images of monomials under the Reynolds operator.%
\begin{theorem}{Theorem}{}{}{subsec-noether-degree-bound-3}%
(Noether):%
\begin{equation*}
R^G = \mathbb{K} [ R_G(\bar x^{\bar \beta}) | \; |\bar \beta| \leq |G|]
\end{equation*}
%
\end{theorem}
\begin{corollary}{Corollary}{}{}{subsec-noether-degree-bound-4}%
(Noether Degree Bound) The ring of invariants is generated in degrees \(\leq |G|\).%
\end{corollary}
Noether's result is a constructive one and provides us with a first computational strategy! We can apply \(R_G\) to all the finitely many monomials in degrees \(\leq |G|\) to get generators for \(R^G\). As the number of monomials grows exponentially with the number of variables and the degree, this is more of a theoretical algorithm, but it does tell us that our goal is at least possible!%
\end{subsectionptx}
%
%
\typeout{************************************************}
\typeout{Subsection 1.2.2 Hilbert Ideal}
\typeout{************************************************}
%
\begin{subsectionptx}{Subsection}{Hilbert Ideal}{}{Hilbert Ideal}{}{}{subsec-hilbert-ideal}
To describe a more sophisticated approach to the search for minimal algebra generators for an invariant ring, we will actually need to consider an ideal instead! Note: for  any \(\{ f_1,..., f_s\} \subseteq R\), the ideal generatd by \(\{f_1,...f_s\}\) and the subalgebra generated by  \(\{f_1,...f_s\}\) over \(\mathbb{K}\) are very different objects.%
\begin{definition}{Definition}{}{subsec-hilbert-ideal-3}%
Let \(J_G := R(R^G)_+\) be the ideal in \(R\) generated by all positive degree invariants. We call \(J_G\) the Hilbert Ideal for this action of \(G\).%
\end{definition}
\begin{theorem}{Theorem}{}{}{subsec-hilbert-ideal-4}%
Let \(J_G \) be the Hilbert ideal in \(R\) for the action of \(G\). If \(J_G = (f_1,...,f_s)\) and every \(f_i\) is invariant so \(f_i\in R^G, \,\, \forall i\), then \(R^G = \mathbb{K}[f_1,...f_s]\)%
\end{theorem}
Note that the condition that every generator is invariant is not hard to satisfy as if you have a generator that is not invariant, then you can apply the Reynolds operator \(R^G\) to obtain a new generator that is. You can now replace the old generator with this new one and still get the same ideal. What is special here is that a set of ideal generators work as algebra generators! Computationally, algebra generators are much harder to find as there is no guarantee to have finitely many of them. However, the Hilbert Basis Theorem tells us that every ideal is finitely generated.%
\end{subsectionptx}
%
%
\typeout{************************************************}
\typeout{Subsection 1.2.3 Presentations}
\typeout{************************************************}
%
\begin{subsectionptx}{Subsection}{Presentations}{}{Presentations}{}{}{subsec-presentations}
When we say that  \(\{f_1,...f_s\} \subseteq R\) are minimal generators for a subring \(S\), we do not exclude the possibility that there is some relation, some polynomial identity, that they satisfy as elements in the bigger ring \(R\). We can describe the relations between the generators via a \terminology{presentation} of the subring.%
\begin{definition}{Definition}{}{subsec-presentations-3}%
Let \(S = \mathbb{K}[f_1,...f_s] \subset R\). A presentation of \(S\) is a map,%
\begin{equation*}
T=: \mathbb{K}[u_1,...u_s] \xrightarrow{\phi}S
\end{equation*}
such that \(\frac{T}{\text{ker}(\phi)} \cong S\). We call the elements of the presentation ideal \(\text{ker}(\phi)\) the syzygies of \(f_i\)'s.%
\end{definition}
Algorithms for finding generators for ideals have been intensely studied and especially in relation with the theory of Groebener bases. We cannot go in the details of these tools, but what is of notice is that they are implemented in Macaulay2 and so we can rely on them in our implementation. In particular, these methods are particularly effective in solving problems in Elimnination Theory.  Often the goal is to compute an ideal of relations hoping that this is less complicated than the original structure, possibly elimnating some variables.%
\begin{proposition}{Proposition}{}{}{subsec-presentations-5}%
(Elimination Theory): In \(S \otimes \mathbb{K}[u_1,...,u_s] = \mathbb{K}[x_1,...,x_n,u_1,...u_s]\) consider the ideal,%
\begin{equation*}
I = (u_i - f_x(\bar x) | \, \left\langle f_i\right\rangle = S
\end{equation*}
Then,%
\begin{equation*}
\text{ker} (\phi)= I \cap \mathbb{K}[u_1,...,u_s]
\end{equation*}
%
\end{proposition}
\begin{algorithm}{Algorithm}{}{}{subsec-presentations-6}%
Compute a Groebner Basis \(G\) for \(I\) with elimination order for the \(x\)'s. Then, \(G \cap \mathbb{K}[y_1,...y_s]\)  is the Groebner Basis for \(ker \phi\)%
\end{algorithm}
\end{subsectionptx}
%
%
\typeout{************************************************}
\typeout{Subsection 1.2.4 Graph of Linear Actions}
\typeout{************************************************}
%
\begin{subsectionptx}{Subsection}{Graph of Linear Actions}{}{Graph of Linear Actions}{}{}{subsec-graph-of-linear-actions}
We can use Elimination Theory to solve our original problem of finding minimal generators for the ring of invariants. We first need to construct a geometric description of the action of a group \(G\).%
\begin{definition}{Definition}{}{subsec-graph-of-linear-actions-3}%
Let \(G\) be a finite matrix group in \(GL_n(\mathbb{K})\). For \(A\in G\) consider,%
\begin{equation*}
V_A = \left\{ ( \bar v, A \bar v) \mid ,v \in V \right \} \subseteq V \oplus V 
\end{equation*}
Then \(A_G = \cup_{A\in G}V_A\) is the subspace arrangement associated to the action of G.%
\end{definition}
Note that \(V_A\) is a linear subspace. So \(\mathbb{I}(V_A)\), the set of polynomials vanishing on \(V_A\), is an ideal generated by linear polynomials, we call this a \terminology{linear ideal}.%
\begin{example}{Example}{}{subsec-graph-of-linear-actions-5}%
Consider%
\begin{equation*}
V_{\begin{pmatrix}
1 \amp 0 \\
0 \amp -1 \\
\end{pmatrix}} = \{(x_1,x_2,x_1,-x_2) \mid x_1, x_2 \in V\} = \mathbb{V}(y_1-x_1, y_2+x_2)
\end{equation*}
%
\end{example}
\end{subsectionptx}
%
%
\typeout{************************************************}
\typeout{Subsection 1.2.5 Subspace Arrangement Approach}
\typeout{************************************************}
%
\begin{subsectionptx}{Subsection}{Subspace Arrangement Approach}{}{Subspace Arrangement Approach}{}{}{subsec-subspace-arrangement-approach}
The finite union of the subspaces \(V_A\), denoted \(\mathcal{A}\) is a subspace arrangement, called the \terminology{group action arrangement}. Via Elimination Theory, we can use the vanishing ideal of \(\mathcal{A}\) to recover the Hilbert Ideal.%
\begin{theorem}{Theorem}{}{}{subsec-subspace-arrangement-approach-3}%
(Derksen): Let \(I_G = \mathbb{I}(A_G) = \cap_{A\in G}\mathbb{I}(V_A) \subseteq \mathbb{K}[x_1,...x_n,y_1,...y_n].\) Then%
\begin{equation*}
(I_G +(y_1,...,y_n)) \cap \R = J_G.
\end{equation*}
%
\end{theorem}
Recent work has shown that the same approach works in the exterior algebra.%
\par
\begin{theorem}{Theorem}{}{}{subsec-subspace-arrangement-approach-5-1}%
(Gandini) Let \(I_G^{'} = \cap_{A\in G} \mathbb{I}(V_A) \subseteq \Lambda(\bar x, \bar y)\). Then%
\begin{equation*}
(I_G^{'} +(y_1,...y_n)) \cap \Lambda(x_1,...,x_n) = J_G^{'} : = \Lambda(\bar x)(\Lambda(\bar x)^G)_+
\end{equation*}
%
\end{theorem}
%
\par
The exterior algebra approach has computational potential. Whilst Derken's approach leads to an algorithm with a long run time, first experiments suggest that a fast algorithm could be implemented for skew polynomials. We aim to pursue this line of inquiry in the near future.%
\end{subsectionptx}
\end{sectionptx}
%
%
\typeout{************************************************}
\typeout{Section 1.3 Specialized algorithms}
\typeout{************************************************}
%
\begin{sectionptx}{Section}{Specialized algorithms}{}{Specialized algorithms}{}{}{sec-}
\begin{introduction}{}%
For some specific types of actions we have faster and simpler algorithms to find invariants.%
\end{introduction}%
%
%
\typeout{************************************************}
\typeout{Subsection 1.3.1 Abelian Groups and Weight Matrices}
\typeout{************************************************}
%
\begin{subsectionptx}{Subsection}{Abelian Groups and Weight Matrices}{}{Abelian Groups and Weight Matrices}{}{}{subsec-AGWM}
Every abelian group \(G\) can be written in its invariant factors decomposition as%
\begin{equation*}
G \cong \mathbb{Z}_{d_1} \oplus....\oplus \mathbb{Z}_{d_r}, 
\end{equation*}
for some unique \(d_i\) such that \(d_i \mid d_{1+1}\) where \(i=1, \ldots , r-1\). In multiplicative notation,%
\begin{equation*}
G \cong \left\langle g_1\right\rangle \oplus...\oplus\left\langle g_r \right\rangle, \,\,\,\,\, |g_i| =d_i.
\end{equation*}
A diagonal action of \(G\) on \(R= \mathbb{K}[\bar x]\) is given by%
\begin{equation*}
g_i \cdot x_j = \mu_i^{w_ij}x_j
\end{equation*}
for \(\mu_i \) a primitive \(d_i^{th}\) root of unity. We can encod the action in the weight matrix%
\begin{equation*}
W = (w_{ij})_{ij} =  
\begin{pmatrix}
\mu_{11} \amp \cdots     \amp   \mu_{1n}   \\
\vdots \amp \ddots \amp      \\
\mu_{r1}  \amp    \cdots     \amp   \mu_{rn}  
\end{pmatrix}
\end{equation*}
where the rows are indexed by the generators \(g_i\) of \(G\) and the columns are indexed by the variables \(x_j\) of \(R\).%
\begin{theorem}{Theorem}{}{}{subsec-AGWM-3}%
A monomial is invariant under the action of \(G\) if an only if its exponent vector is in the kernel of the weigtht matrix \(W \). In symbols,%
\begin{equation*}
\bar x^{\bar \beta} \in R^G \iff W \bar \beta \cong (0,...,0),
\end{equation*}
where the entry in position \(i\) is computed modulo \(d_i\).%
\end{theorem}
Interpreting each row has the weight of the action of the generator \(g_i\), we have that \(g_i\) acts trivially on the monomial \(\bar x^{\bar \beta}\) precisely when%
\begin{equation*}
\mu_i^{\sum_j w_{ij} \beta_j} = 1
\end{equation*}
so \((W \bar \beta)_i =0\) modulo \(d_i\) as \(\mu_i\) is a primitve \(d_i^{th}\) root of unity.%
\par
We can use this result computationally. As the action is diagonal, the invariant ring is generated by monomials. By Noether Degree Bound we only need to examine all monomials of degree less than the order of the group \(G\). Then, by the theorem above, if we can sort the monomials in terms of their weigtht \(W\bar \beta\), then the monomials with weight \(\bar 0\) will be invariant.%
\end{subsectionptx}
%
%
\typeout{************************************************}
\typeout{Subsection 1.3.2 Permutation actions}
\typeout{************************************************}
%
\begin{subsectionptx}{Subsection}{Permutation actions}{}{Permutation actions}{}{}{subsec-orbitsums}
The symmetric group \(S_n \) acts naturally on \(\{1, ... , n\}\) by permuting its elements. This action allows us to define a representation \(V \cong \mathbb{K}^n\) where on the basis vectors \(\{e_1, ... , e_n\}\), the permutation \(\sigma\) acts by \(\sigma \cdot e_i = e_{\sigma(i)}\). As \(S_n \) acts on \(V\) by permuting its basis vectors, we have that \(V\) is a permutation representation induced by the permutation action of \(S_n\).%
\begin{example}{Example}{}{subsec-orbitsums-3}%
Consider the group \(S_4\). The matrix for the action of \((1 \,2\,3\,4) \) is given by%
\begin{equation*}
M_{(1 \,2\,3\,4)} =  \begin{pmatrix}
0 \amp 1 \amp 0 \amp 0 \\
0 \amp 0 \amp 1 \amp 0 \\
0 \amp 0 \amp 0 \amp 1 \\
1 \amp 0 \amp 0 \amp 0 
\end{pmatrix}   
\end{equation*}
and then we have it acting on a vector by matrix multiplication,%
\begin{equation*}
\begin{pmatrix}
0 \amp 1 \amp 0 \amp 0 \\
0 \amp 0 \amp 1 \amp 0 \\
0 \amp 0 \amp 0 \amp 1 \\
1 \amp 0 \amp 0 \amp 0 
\end{pmatrix} \begin{pmatrix}
v_1 \\
v_2 \\
v_3\\
v_4
\end{pmatrix}   =    \begin{pmatrix}
v_4 \\
v_1 \\
v_2\\
v_3
\end{pmatrix}
\end{equation*}
%
\end{example}
Now consider the action of the symmetric group \(S_n\) on the polynomial ring \(R=\mathbb{K}[x_1,x_2,...,x_n]\). The symmetric polynomials are the invariant polynomials under this action..%
\begin{definition}{Definition}{}{def-symmetricpolynomial}%
We call \(f \in R\), a symmetric polynomial if%
\begin{equation*}
f(x_1,x_2,...,x_n) = f(x_{\sigma(1)},x_{\sigma(2)},...,x_{\sigma(n)})
\end{equation*}
for all permutations of \(\sigma \in S_n\).%
\end{definition}
An example of symmetric polynomials is given by the \terminology{elementary symmetric polynomials}.%
\begin{definition}{Definition}{}{def-elemsymm}%
The elementary symmetric polynomials \(e_0,e_1,...,e_n\) in \(\mathbb{K}[x_1,...,x_n]\) are defined by%
\begin{equation*}
e_{m}=\sum_{|J|=m} \bar x_J = \sum_{j_1 \lt j_2 \lt ... \lt j_m} x_{j_1}x_{j_2}...x_{j_m}, 
\end{equation*}
with \(e_0=1\).%
\end{definition}
More generally, we can consider a permutation action by some subgroup \(G\) of \(S_n\). For any monomial in \(R\), we can consider its orbit under the action of \(G\).%
\begin{definition}{Definition}{}{def-orbit}%
The \(G-orbit\) of any element \(f \in R\) is%
\begin{equation*}
\text{orb}(f) = \{g \cdot f \mid g \in G\}\text{.}
\end{equation*}
%
\end{definition}
Any permutation in \(G\) acts on the orbit \(\text{orb}(f)\) by rearranging its elements. As a result, the orbit sums will give us invariant polynomials.%
\begin{definition}{Definition}{}{def-orbitsum}%
The orbit sum of \(f\) the sum of over the elements of  \(\text{orb}(f)\).%
\end{definition}
We can find a set of minimal invariants by computing all orbit sums of all monomials of degree \(\lt |G|\). But this is computationally expensive and will lead to a lot of redundant computations. Instead, we will use a result that tells us that we only need to compute the orbit sums of some special monomials. Consider the exponent sequence \(\beta\) of a monomial \(\bar x^{\bar \beta}\) and rearrange it to obtain an integer partition \(\lambda\), where \(\lambda_1 \gt \lambda_2 \gt ... \gt \lambda_m\)%
\begin{definition}{Definition}{}{def-specialmonomials}%
We call a monomial \terminology{special} if the entries in its associated partition decrease by at most one at each step and the last entry is 0.%
\end{definition}
The definition of special depends on the number of variables or the number of parts in the integer partition. So \(x_1^n x_2^{n-1}....x_{n-1}^1 x_n^0\) would not be special within \(\mathbb{K}[x_1, ... , x_{n-1}]\), but it is special in \(\mathbb{K}[x_1, ... , x_{n}]\).%
\begin{theorem}{Theorem}{}{}{thm-gobel}%
(Göbel) Let \(\phi:G \mapsto \text{GL}(n,\mathbb{K})\) be a permutation representation of a finite group acting on \(R = \mathbb{K}[x_1,...,x_n]\). Then the ring of invariants \(R^G\) is generated as an algebra by the top elementary symmetric function \(e_n = x_1...x_n\) and the orbit sum of the special monomials.%
\end{theorem}
\begin{corollary}{Corollary}{}{}{cor-gobel}%
(Göbel's Bound) In a permutation representation of dimension at least 3, the degree of a minimal generating invariant is at most \(\binom{n}{2}\).%
\end{corollary}
We have that the bound follows from the theorem as the degree of a special momial is at most \(\binom{n}{2}\) and this is larger than \(n\), the degree of \(e_n\), when \(n \geq 3\).%
\par
To show that orbits of special monomial generate the ring of invariants, one needs to consider a reduction argument. If you start with a non-special monomial, the entries of its partition are not decreasing by at most 1 and we call the first place where there is a jump a \(gap\) in the partition. Starting from a non-special monomial we can obtain a reduced monomial by decreasing all the largest exponents up to the gap by one. The reduced monomial will be closer to being special as the gap itself will be reduced by 1.%
\begin{example}{Example}{}{subsec-orbitsums-19}%
\(\mathbb{K}[x_1,x_2,x_3,x_4]\)\(x_1^2x_2^4x_3 \)\((4,2,1)\)%
\begin{equation*}
x_1^2x_2^4x_3 \mapsto x_1^2x_2^3 x_3,
\end{equation*}
\end{example}
Algorithmically, we can reduce any monomial to a special one by reducing the upper degrees repeatedly until the monomial is special. So the general idea of the proof of Gobel's theorem is to show that the orbit sum of any monomial can be rewritten as a sum of orbit sums of special monomials or special monomials times some power of \(e_n\). In our ongoing work we are using Gobel's theorem as a tool for reducing the complexity of computing invariants for permutation actions.%
\end{subsectionptx}
\end{sectionptx}
%
%
\typeout{************************************************}
\typeout{Section 1.4 InvariantRings package}
\typeout{************************************************}
%
\begin{sectionptx}{Section}{InvariantRings package}{}{InvariantRings package}{}{}{sec-invariantrings-packages}
\begin{introduction}{}%
We conclude with references to the algorithms implemented in the InvariantRing package and examples of its implementation. Version 2.0 of this package was accepted for publication in volume 14 of Journal of Software for Algebra and Geometry on 2023-09-14, in the article The InvariantRing package for Macaulay2 (DOI: 10.2140\slash{}jsag.2024.14.5). That version can be obtained from the journal or from the Macaulay2 source code repository.%
\end{introduction}%
%
%
\typeout{************************************************}
\typeout{Subsection 1.4.1 References for the implemented algorithms}
\typeout{************************************************}
%
\begin{subsectionptx}{Subsection}{References for the implemented algorithms}{}{References for the implemented algorithms}{}{}{subsec-references}
%
\begin{itemize}[label=\textbullet]
\item{}An elimination theory algorithm that computes the Hilbert ideal for any linearly reductive group: Derksen, H. and Kemper, G. (2015). Computational Invariant Theory. Heidelberg: Springer. Algorithm 4.1.9, pp 159-164%
\item{}A simple and efficient algorithm for invariants of tori based on: Derksen, H. and Kemper, G. (2015). Computational Invariant Theory. Heidelberg: Springer. Algorithm 4.3.1 pp 174-177%
\item{}An adaptation of the tori algorithm for invariants of finite abelian groups based on: Gandini, F. Ideals of Subspace Arrangements. Thesis (Ph.D.)-University of Michigan. 2019. ISBN: 978-1392-76291-2. pp 29-34.%
\item{}King's algorithm and the linear algebra method for invariants of finite groups: Derksen, H. and Kemper, G. (2015). Computational Invariant Theory. Heidelberg: Springer. Algorithm 3.8.2, pp 107-109; pp 72-74%
\item{}The algorithms for primary and secondary invariants, and Molien series of finite groups implemented in version 1.1.0 of this package by: Hawes, T. Computing the invariant ring of a finite group. JSAG, Vol. 5 (2013). pp 15-19. DOI: 10.2140\slash{}jsag.2013.5.15%
\end{itemize}
The orbit sum approach is under development and will be released with version 3.0 of the package. Our implementation is following the description by Mara D. Neusel, in her book Invariant Theory from the AMS The Student Mathematical Library (2007, Volume 36), DOI: \href{http://dx.doi.org/10.1090/stml/036}{\nolinkurl{http://dx.doi.org/10.1090/stml/036}}%
\end{subsectionptx}
%
%
\typeout{************************************************}
\typeout{Subsection 1.4.2 InvariantRing Library Demos}
\typeout{************************************************}
%
\begin{subsectionptx}{Subsection}{InvariantRing Library Demos}{}{InvariantRing Library Demos}{}{}{sec-invariantring-demos}
The InvariantRing package in Macaulay2 provides tools to study and compute invariant rings of group actions. To get started, install the package.%
\begin{sageinput}
installPackage "InvariantRing"
\end{sageinput}
\begin{example}{Example}{Special Linear Group Actions on two variables.}{sec-invariantring-demos-4}%
A classical example: the standard action of \(\text{SL}_2\) on \(\mathbb{Q}^2\). The ring R carries a linearly reductive action defined via the matrix SL2std. The invariants and Hilbert ideal are then computed.%
\begin{sageinput}
restart
needsPackage "InvariantRing"
B = QQ[a,b,c,d]
A = ideal(a*d - b*c - 1)
SL2std = matrix{{a,b},{c,d}}
R = QQ[x_1..x_2]
V = linearlyReductiveAction(A,SL2std,R) 
invariants V
elapsedTime hilbertIdeal V
\end{sageinput}
\end{example}
\begin{example}{Example}{Diagonal Actions of Abelian Groups.}{sec-invariantring-demos-5}%
This example demonstrates a diagonal action of the abelian group \(C_3 \times C_3\) on a polynomial ring. After defining the diagonal weights, we compute the invariant ring and its Hilbert series.%
\begin{sageinput}
restart
needsPackage "InvariantRing"
R = QQ[x_1..x_3]
W = matrix{{1,0,1},{0,1,1}}
L = {3,3}
T = diagonalAction(W,L,R)
S = R^T
invariantRing T
I = definingIdeal S
Q = ring I
F = res I
hilbertSeries S
equivariantHilbertSeries T
\end{sageinput}
\end{example}
\begin{example}{Example}{Linearly Reductive Actions: Permutations and Binary Forms.}{sec-invariantring-demos-6}%
Here's how the symmetric group \(S_2\) acts via a matrix of projection operators. We identify which polynomials are invariant under the group action.%
\begin{sageinput}
restart
needsPackage "InvariantRing"
S = QQ[z]
A = ideal(z^2 - 1)
M = matrix{{(1+z)/2, (1-z)/2},{(1-z)/2,(1+z)/2}}
R = QQ[a,b]
X = linearlyReductiveAction(A,M,R)
isInvariant(a,X)
invariants X
\end{sageinput}
Now we compute the invariants of binary quadratics and quartics using \(\text{SL}_2\) actions. These involve basis substitutions in a ring of forms and are more computationally demanding.%
\begin{sageinput}
restart
needsPackage "InvariantRing"
S = QQ[a,b,c,d]
I = ideal(a*d - b*c - 1)
A = S[u,v]
M = transpose (map(S,A)) last coefficients sub(basis(2,A),{u=>a*u+b*v,v=>c*u+d*v})
R = QQ[x_1..x_3]
L = linearlyReductiveAction(I,M,R)
hilbertIdeal L
invariants L
invariants(L,4)
invariants(L,5)
\end{sageinput}
\begin{sageinput}
restart
needsPackage "InvariantRing"
S = QQ[a,b,c,d]
I = ideal(a*d - b*c - 1)
A = S[u,v]
M4 = transpose (map(S,A)) last coefficients sub(basis(4,A),{u=>a*u+b*v,v=>c*u+d*v})
R4 = QQ[x_1..x_5]
L4 = linearlyReductiveAction(I,M4,R4)
elapsedTime hilbertIdeal L4
elapsedTime X = invariants L4
g2 = X_0/12
g3 = -X_1/216
256*(g2^3 - 27*g3^2)
1728*(g2^3)/(g2^3 - 27*g3^2)
\end{sageinput}
\end{example}
\begin{example}{Example}{Matrix Invariants and Conjugation Actions.}{sec-invariantring-demos-7}%
We define SL₂ actions on 2\(\times\)2 and 3\(\times\)3 matrices of binary or ternary forms. The conjugation action creates sophisticated invariants under change of basis.%
\begin{sageinput}
restart
needsPackage "InvariantRing"
S = QQ[g_(1,1)..g_(2,2),t]
I = ideal((det genericMatrix(S,2,2))*t-1)
Q = S/I
A = Q[y_(1,1)..y_(2,2)]
Y = transpose genericMatrix(A,2,2)
g = promote(genericMatrix(S,2,2),A)
G = reshape(A^1,A^4,g*Y*inverse(g)) // (vars A)
G = lift(map(A^4,A^4,G),S)
R = QQ[x_(1,1)..x_(2,2)]
L = linearlyReductiveAction(I,G,R)
elapsedTime H=hilbertIdeal(L)
elapsedTime invariants L
\end{sageinput}
The same process is repeated for 3\(\times\) 3 matrices. This involves 9-dimensional vector spaces and is more computationally demanding.%
\begin{sageinput}
restart
needsPackage "InvariantRing"
S = QQ[g_(1,1)..g_(3,3),t]
I = ideal((det genericMatrix(S,3,3))*t-1)
Q = S/I
A = Q[y_(1,1)..y_(3,3)]
Y = transpose genericMatrix(A,3,3)
g = promote(genericMatrix(S,3,3),A)
G = reshape(A^1,A^9,g*Y*inverse(g)) // (vars A)
G = lift(map(A^9,A^9,G),S)
R = QQ[x_(1,1)..x_(3,3)]
L = linearlyReductiveAction(I,G,R)
elapsedTime H=hilbertIdeal(L)
elapsedTime invariants(L,1)
elapsedTime invariants(L,2)
elapsedTime invariants(L,3)
\end{sageinput}
\end{example}
\begin{example}{Example}{Finite Group Actions.}{sec-invariantring-demos-8}%
Finally, we examine the symmetric group \(S_4\) acting on 4 variables. The current version of the package can use both King's algorithm and a slower linear algebra method to compute primary and secondary invariants. Future methods using Gobel's theorem should improve on the speed of this computation.%
\begin{sageinput}
restart
needsPackage "InvariantRing"
R = QQ[x_1..x_4]
L = apply({[2,1,3,4],[2,3,4,1]},permutationMatrix);
S4 = finiteAction(L,R)
elapsedTime invariants S4
elapsedTime invariants(S4,Strategy=>"LinearAlgebra")
elapsedTime p=primaryInvariants S4
elapsedTime secondaryInvariants(p,S4)
elapsedTime hironakaDecomposition(S4)
\end{sageinput}
\end{example}
\end{subsectionptx}
\end{sectionptx}
\end{chapterptx}
\end{document}
